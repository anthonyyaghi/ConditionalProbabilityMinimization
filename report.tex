\documentclass[conference]{IEEEtran}

% *** CITATION PACKAGES ***
%
\usepackage{cite}
% cite.sty was written by Donald Arseneau
% V1.6 and later of IEEEtran pre-defines the format of the cite.sty package
% \cite{} output to follow that of IEEE. Loading the cite package will
% result in citation numbers being automatically sorted and properly
% "compressed/ranged". e.g., [1], [9], [2], [7], [5], [6] without using
% cite.sty will become [1], [2], [5]--[7], [9] using cite.sty. cite.sty's
% \cite will automatically add leading space, if needed. Use cite.sty's
% noadjust option (cite.sty V3.8 and later) if you want to turn this off
% such as if a citation ever needs to be enclosed in parenthesis.
% cite.sty is already installed on most LaTeX systems. Be sure and use
% version 4.0 (2003-05-27) and later if using hyperref.sty. cite.sty does
% not currently provide for hyperlinked citations.
% The latest version can be obtained at:
% http://www.ctan.org/tex-archive/macros/latex/contrib/cite/
% The documentation is contained in the cite.sty file itself.

\ifCLASSINFOpdf
 \usepackage[pdftex]{graphicx}
 \graphicspath{{pics/}}
 \DeclareGraphicsExtensions{.pdf,.jpeg,.png}
\fi
 
\begin{document}
\title{Solving complex reliability networks using conditional probability.}

\author{\IEEEauthorblockN{Anthony Yaghi}
\IEEEauthorblockA{School of Electrical and Computer Engineering\\
Lebanese American University\\
Lebanon, Byblos\\
Email: anthony.yaghi@lau.edu}
}

\maketitle

\begin{abstract}

\end{abstract}

\section{Introduction}
The field of reliability study started to take shape during the second world
war. A lot of new military equipments were introduced and the main appplication
of reliability study was focused on the vaccum tube which was an integral part of many
of these equipements, like radar systems. And so, the military needed a
measurement of how likely these costly equipements were to fails in order to
mitigate the risk of a malfunction during the war. Today, reliability study
is a large field with both military and commercial applications. A common way to
represent a system is using a graph or a network where each edge represents a
component, by using this representation, one can visualize the system which
makes it easier to apply different evaluation methods. One such evaluation
method is the conditional probability approach, it consists of simplifying the
system by reducing it into a set of series-parallel systems and re-combining
their solution. The conditional probability approach is a very powerful tool
that can be used to solve any complex system elegantly; in this project we will
explore the possibility of implementing this method on a computer. The objective
is to have a working implementation that is capable of taking a representation
of a system and finding the best way to reduce it using the conditional probability
approach. In addition, the implementation could be used as an educational tool
to show visually how a system gets divided into 2 new sub-systems for a given
component. We will start by explaining the conditional probability
method and how to transform it into an algorithm. In the second part of the
report, we will present the actual implementation and the technical challlenges
faced. Finally, we will see how the implementation will perform on a set of
different problems/networks.\\

\section{The Conditional Probability Approach}
The conditional probability approach consists of gradually breaking a complex
system into smaller and simpler series-parallel sub-systems.



\section{Conclusion}
The conclusion goes here.


\begin{thebibliography}{1}

\bibitem{IEEEhowto:kopka}
H.~Kopka and P.~W. Daly, \emph{A Guide to \LaTeX}, 3rd~ed.\hskip 1em plus
  0.5em minus 0.4em\relax Harlow, England: Addison-Wesley, 1999.

\end{thebibliography}

\end{document}